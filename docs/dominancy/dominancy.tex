\documentclass[a4paper,12pt]{article}

% Set margins
\usepackage[hmargin=2.5cm, vmargin=3cm]{geometry}

\frenchspacing

% Language packages
\usepackage[utf8]{inputenc}
\usepackage[T1]{fontenc}
% \usepackage[magyar]{babel}

% AMS
\usepackage{amssymb,amsmath}

% Graphic packages
\usepackage{graphicx}

% Colors
\usepackage{color}
\usepackage[usenames,dvipsnames]{xcolor}

% Enumeration
\usepackage{enumitem}

% Links
\usepackage{hyperref}

% Question
\newenvironment{question}[1]
{\noindent\textcolor{OliveGreen}{$\circ$ \textit{#1}}

\smallskip

\color{Gray}

}{\bigskip}

% Task
\newenvironment{task}[1]
{\noindent\textcolor{RoyalBlue}{$\circ$ \textit{#1}}

\smallskip

\color{Gray}

}{\bigskip}

% Notification
\newenvironment{notification}[1]
{\noindent\textcolor{Peach}{$\circ$ \textit{#1}}

\smallskip

\color{Gray}

}{\bigskip}

% Problem
\newenvironment{problem}[1]
{\noindent\textcolor{OrangeRed}{$\circ$ \textit{#1}}

\smallskip

\color{Gray}

}{\bigskip}

% Solution
\newenvironment{solution}
{\noindent\color{Violet}}{\bigskip}

% Starred
\newenvironment{starred}
{\noindent\color{Maroon}}{\bigskip}

% QUEST: How can we guarantee that there is no contradiction in the rulebase (with negated symbols)?

\begin{document}

\begin{center}
    \Large \textbf{Fuzzy Rule Interpolation with Dominancy}
\end{center}

\section{Introduction}

The \textit{Fuzzy Rule Interpolation} is an intuitive method for describing fuzzy rulebases in continuous space. The presented concept introduces the dominancy as the way of handling contradictionary cases.

\section{Rule Surfaces}

Let define the space of \textit{antecedent dimensions}. We assume that we have $n$ antecedents, which are real-valued. In this case the search space is $\mathbb{R}^n$. Let denote $r_i \in \mathbb{R}^n$ a rule, which actually a point of the space.

We use a further, so called \textit{consequent dimension}. It also has real values. We assign a consequent value for any rule points. This value at the given point shows the response/output of the system. We can allow lower dimension rules also.

Our goal is to define a rule surface which able to \textit{interpolate} the consequent value between the defined rule points.

\begin{question}{Have we consider separately the extrapolation case?}
\end{question}

\section{Contradictions}

The rule surface is an $\mathbb{R}^n \rightarrow \mathbb{R}$ mapping. In the assumption of interpolation, we cannot use different consequent values for the same antecedent value.
\begin{itemize}
\item It can be checked easily, when all of the rules have $n$ dimensions.
\item The checking of lower dimensional rules are more time consuming, but the calculation is straightforward.
\end{itemize}

We would like to hold that all the defined rule antecedents result the defined consequents. We would like to resolve the contradictions by introducing the concept of \textit{dominancy}.

\section{Distances and Weights}

We can calculate the distance of the observation and a rule. Let $x \in \mathbb{R}^n$ the observation and $r \in \mathbb{R}$ a rule. Let denote $d_j$ the distance of the rule and the observation according the the $j$-th antecedent dimension. Therefore we can calculate the rule distance as
$$
\delta(x, r) =
\dfrac{
	\sqrt{\displaystyle \sum_{j=1}^{k} d_{j}^{2}}
}{
	\sqrt{k}
}.
$$
(The $k$ is the number of antecedent dimensions in the rule, $k \leq n$.)

We define the weight as
$$
w(x, r) = \dfrac{1}{\delta(x, r)}.
$$

\begin{question}{Can we avoid somehow the consideration of infinit values?}
\end{question}

\section{Dominancy}

The dominancy is a relation between the rulebases. We have to distinguish the \textit{dominating} and the \textit{dominated} rulebase. (All rulebase can have at most 1 dominant rulebase, which results a tree structure.)

\begin{verbatim}
rulebase "A"
    ...
end

rulebase "B"
   dominated by "A"
   ...
end
\end{verbatim}

In this case the rulebase $B$ has dominated by the rulebase $A$.

Let denote $a$ the consequence of $A$ and $b$ is the consequence of $B$. Let $d$ the aggregated (in simple case the minimal) normalized distance of the observation from the rules of the dominant rulebase. We can calculate the consequence as
$$
c = (1 - d) \cdot a + d \cdot b.
$$
As we can see, it results $a$ when the dominant rulebase matches. The normalization guarantees that the maximal distance is 1. In that case the result will be $b$ (which means that the dominant rulebase does not affect the result).

\begin{question}{Should we consider all of the dominant rules or just the closest one?}
\end{question}

\section{Evaluation}

For evaluating the dominant rulebase, we have to consider the followings.
\begin{itemize}
\item We have to check that there is no circle in the dominancy graph.
\item We must evaluate the most dominant rulebases first.
\end{itemize}

\end{document}
