\documentclass[a4paper,12pt]{article}

% Set margins
\usepackage[hmargin=2.5cm, vmargin=3cm]{geometry}

\frenchspacing

% Language packages
\usepackage[utf8]{inputenc}
\usepackage[T1]{fontenc}
% \usepackage[magyar]{babel}

% AMS
\usepackage{amssymb,amsmath}

% Graphic packages
\usepackage{graphicx}

% Colors
\usepackage{color}
\usepackage[usenames,dvipsnames]{xcolor}

% Enumeration
\usepackage{enumitem}

% Links
\usepackage{hyperref}

% Question
\newenvironment{question}[1]
{\noindent\textcolor{OliveGreen}{$\circ$ \textit{#1}}

\smallskip

\color{Gray}

}{\bigskip}

% Task
\newenvironment{task}[1]
{\noindent\textcolor{RoyalBlue}{$\circ$ \textit{#1}}

\smallskip

\color{Gray}

}{\bigskip}

% Notification
\newenvironment{notification}[1]
{\noindent\textcolor{Peach}{$\circ$ \textit{#1}}

\smallskip

\color{Gray}

}{\bigskip}

% Problem
\newenvironment{problem}[1]
{\noindent\textcolor{OrangeRed}{$\circ$ \textit{#1}}

\smallskip

\color{Gray}

}{\bigskip}

% Solution
\newenvironment{solution}
{\noindent\color{Violet}}{\bigskip}

% Starred
\newenvironment{starred}
{\noindent\color{Maroon}}{\bigskip}

% QUEST: How can we guarantee that there is no contradiction in the rulebase (with negated symbols)?

\begin{document}

\begin{center}
    \Large \textbf{Dominancy}
\end{center}

\section{Introduction}

We use \textit{conjunction between predicates} and \textit{disjunction between rules}.

We would like to extend the expressive power of the behavior description language by using negation. The fundamental problem is that in the case of multi-valued logic the negation cannot be defined.

\section{Rule surface calculation}

\section{One-dimensional case}

We assume that
\begin{itemize}
\item the consequent value of the bounds of the universe has defined with rules, and
\item there is no contradiction in the rulebases.
\end{itemize}
These are necessary for the interpolation. The rules at the bounds makes the default values optional, because any inner point can be interpolated from the neightbour rules.

From these reasons we have to use at least two rules on any universe. For instance, let consider a simple rulebase $\left\{ x = 0 \Rightarrow c = 0, x = 1 \Rightarrow c = 1 \right\}$. We would like to express the $x \neq 0$ and $x \neq 1$ relations. We would like to avoid the negation on the antecedent side. Instead, we use a default value for cases, where the rule does not match. (In the case of interpolation it does not require exact matching.) For making the rulebase consistent we have to use the same default value for $x \neq 0$ and $x \neq 1$. From illustrative reasons, we choose $2$ as the default value.

The exact shape of the rulebase \textit{surface} (currently a curve) depends on the interpolation function. In general, we can say that $c(0.5) \in [0.5, 2]$.

For any higher dimension we have to use the same default value for the rulebase. It is necessary, for avoiding contradictions. For example $c(0.2) = 2$ and $c(0.2) = 3$ in this case.

Therefore, the dominated surface must be constant in the case of negation, for avoiding contradictions.

Unfortunately, the sample rulebase has also contains contradiction, for example $c(0) = 0$ (according to $x = 0 \Rightarrow c = 0$), but $c(0) = 2$ (according to $x \neq 1 \Rightarrow c = 2$). Symmetrically, $c(1) = 1$ (according to $x = 1 \Rightarrow c = 1$), but $c(1) = 2$ (according to $x \neq 0 \Rightarrow c = 2$).

\section{Two-dimensional case}

Commutativity of dominancy application

De Morgan rule

\section{Three-dimensional case}

\end{document}
